% Author: Dominik Harmim <xharmi00@stud.fit.vutbr.cz>

\documentclass[a4paper, 11pt]{article}

\usepackage[czech]{babel}
\usepackage[utf8]{inputenc}
\usepackage[left=2cm, top=3cm, text={17cm, 24cm}]{geometry}
\usepackage{times}
\usepackage{hyperref}
\hypersetup{colorlinks = true}

\begin{document}

	%%%%%%%%%%%%%%%%%%%%%%%%%%%%%%%% Uvodni stranka %%%%%%%%%%%%%%%%%%%%%%%%%%%%%%%%
	\begin{titlepage}
		\begin{center}
			\Huge
			\textsc{Vysoké učení technické v~Brně} \\
			\huge
			\textsc{Fakulta informačních technologií} \\
			\vspace{\stretch{0.382}}
			\LARGE
			Typografie a~publikování\,--\,4.~projekt \\
			\Huge
			Bibliografické citace
			\vspace{\stretch{0.618}}
		\end{center}

		{\Large
			\today
			\hfill
			Dominik Harmim
		}
	\end{titlepage}


	%%%%%%%%%%%%%%%%%%%%%%%%%%%%%%%% LaTeX %%%%%%%%%%%%%%%%%%%%%%%%%%%%%%%%
	\section{Systém \LaTeX}

	\subsection{Definice}
	\LaTeX\ je komplexní sada příkazů, využívající propracovaný sázecí program \TeX, pro přípravu široké
	škály dokumentů, od vědeckých článků, zpráv, prezentací, až po celé knihy. \TeX\ i~samotný \LaTeX\ je
	otevřený software, k~dispozici zcela zdarma. Dokonce je možné tento sofware upravovat a~distribuovat
	ho nebo jeho části pod jiným jménem, aby nedošlo k~záměně. O~podrobnějších rozdílech mezi systémy
	\TeX\ a~\LaTeX\ pojednává \cite{Kopkac2004}.


	\subsection{Práce s~{\LaTeX}em}
	Práce s~celým systémem připomíná programování, protože sestává z~těchto tří kroků, jak je uvedeno
	v~\cite{Rybicka2003}:
	\begin{enumerate}
		\item psaní (úprava) zdrojového textu,
		\item překlad\,--\,vysázení,
		\item prohlížení.
	\end{enumerate}
	Podrobnější informace o~tom, jak probíhá překlad zdrojových souborů do výsledného dokumentu se lze
	dočíst v~\cite{Rybicka2003}.


	\subsection{Proč používat \LaTeX}
	\LaTeX\ vyniká obrovskou přesností. Pro rozsáhlé dokumenty se spoustou matematických vzorců a~jiných symbolů,
	číslovanými obrázky, tabulkami a~odkazy na ně v~textu je výhodnější použít \LaTeX\ viz \cite{Martinek2010}.

	Článek \cite{Simecek2013} uvádí, že vlastnosti {\LaTeX}u můžeme shrnout do několika slov: jednoduchost,
	elegance a~možnost být kreativní.


	\subsection{Struktura dokumentu}
	Dokument v~{\LaTeX}u je rozdělen na dvě části. První část (preambule) obsahuje globální nastavení dokumentu.
	Druhá část už obsahuje vlastní text viz \cite{Svamberg2001}.


	\subsection{Online editory pro \LaTeX}
	Nevýhodou {\LaTeX}u je skutečnost, že není tak rozšířený a dostupný na všech počítačích, jako např. MS Word
	nebo jiné běžně dostupné editory. Díky online editorům pro \LaTeX\ můžeme tuto nevýhodu minimalizovat. Příkladem
	těchto online editorů jsou {\LaTeX}lab, Tex-On-Web, aj. viz \cite{Sokol2012}.

	Existují dokonece i~nástroje pro převod dokumentů z~aplikace MS Word do {\LaTeX}u. Této problematice se věnuje
	\cite{Simek2009}.


	\subsection{Matematická sazba}
	Jedna z~nejsilnějších stránek {\LaTeX}u je sazba matematického textu. \LaTeX\ tvoří vzorečky ve vnitřním
	matematickém módu \$ \ldots \$ nebo v~display módu \$\$ \ldots \$\$. V~tomto módu lze velmi jednoduše
	sázet zlomky, matematické symboly a~znaky, závorky a~další speciality viz \cite{Olsak2014}. Existují
	dokonce i~nástroje, které z~ručně psaného textu dokáží vygenerovat zdrojový kód pro \LaTeX\
	\cite{Oksuz2008}.

	Příklad vysázené rovnice (lze nalézt v~\cite{CazarezCastro2012}):
	$$
		\eta = -2j \frac{1 - \mathrm{exp}\{ - \frac{\Delta q}{j} \}}{1 + \mathrm{exp}\{ - \frac{\Delta q}{j} \}}
	$$


	%%%%%%%%%%%%%%%%%%%%%%%%%%%%%%%% Citace %%%%%%%%%%%%%%%%%%%%%%%%%%%%%%%%
	\newpage
	\bibliographystyle{czechiso}
	\renewcommand{\refname}{Literatura}
	\bibliography{proj4}

\end{document}
