\documentclass[a4paper, 11pt, twocolumn]{article}

% Dominik Harmim <xharmi00@stud.fit.vutbr.cz>

\usepackage[czech]{babel}
\usepackage[utf8]{inputenc}
\usepackage[left=1.5cm, top=2.5cm, text={18cm, 25cm}]{geometry}
\usepackage{times}
\usepackage{amsthm, amssymb, amsmath}

\theoremstyle{definition}
\newtheorem{definition}{Definice}[section]

\theoremstyle{plain}
\newtheorem{algorithm}[definition]{Algoritmus}
\newtheorem{sentence}{Věta}

\begin{document}

	\begin{titlepage}
		\begin{center}
			{\Huge\textsc{
				Fakulta informačních technologií \\
				Vysoké učení technické v~Brně
			}}
			\vspace{\stretch{0.382}}
			{\LARGE
				\\ Typografie a~publikování -- 2. projekt \\
				Sazba dokumentů a~matematických výrazů
			}
			\vspace{\stretch{0.618}}
		\end{center}

		{\Large
			\the\year
			\hfill
			Dominik Harmim
		}
	\end{titlepage}


	\section*{Úvod}

	V~této úloze si vyzkoušíme sazbu titulní strany, matematických vzorců, prostředí a~dalších
	textových struktur obvyklých pro technicky zaměřené texty, například rovnice (\ref{eq_1}) nebo definice
	\ref{definition_bezkontextova_gramatika} na straně \pageref{definition_bezkontextova_gramatika}.

	Na titulní straně je využito sázení nadpisu podle optického středu s~využitím zlatého řezu.
	Tento postup byl probírán na přednášce.


	\section{Matematický text}

	Nejprve se podíváme na sázení matematických symbolů a~výrazů v~plynulém textu. Pro množinu
	$ V $ označuje $ \text{card}(V) $ kardinalitu $ V $. Pro množinu $ V $ reprezentuje $ V^* $
	volný monoid generovaný množinou $ V $ s~operací konkatenace. Prvek identity ve volném monoidu
	$ V^* $ značíme symbolem $ \varepsilon $. Nechť $ V^+ = V^* - \{ \varepsilon \} $. Algebraicky
	je tedy $ V^+ $ volná pologrupa generovaná množinou $ V $ s~operací konkatenace. Konečnou
	neprázdnou množinu $ V $ nazvěme \emph{abeceda}. Pro $ w \in V^* $ označuje $ |w| $ délku řetězce
	$ w $. Pro $ W \subseteq V $ označuje $ \text{occur}(w, W) $ počet výskytů symbolů z~$ W $
	v~řetězci $ w $ a~$ \text{sym}(w, i) $ určuje $ i $-tý symbol řetězce $ w $; například
	$ \text{sym}(abcd, 3) = c $.

	Nyní zkusíme sazbu definic a~vět s~využitím balíku \texttt{amsthm}.

	\begin{definition}
		\label{definition_bezkontextova_gramatika}
		\emph{Bezkontextová gramatika} je čtveřice $ G = (V, T, P, S) $, kde $ V $ je totální abeceda,
		$ T \subseteq V $ je abeceda terminálů, $ S \in (V - T) $ je startující symbol a~$ P $ je
		konečná množina \emph{pravidel} tvaru $ q \, {:} \: A \rightarrow \alpha $, kde  $ A \in (V - T) $,
		$ \alpha \in V^* $ a~$ q $ je návěští tohoto pravidla. Nechť $ N = V - T $ značí abecedu neterminálů.
		Pokud $ q \, {:} \: A \rightarrow \alpha \in P $, $ \gamma, \delta \in V^* $, $ G $ provádí derivační
		krok z~$ \gamma{A}\delta $ do $ \gamma\alpha\delta $ podle pravidla
		$ q \, {:} \: A \rightarrow \alpha $, symbolicky píšeme
		$ \gamma{A}\delta \Rightarrow \gamma\alpha\delta \hspace{0.5em} [q \, {:} \: A \rightarrow \alpha] $
		nebo zjednodušeně $ \gamma{A}\delta \Rightarrow \gamma\alpha\delta $. Standardním způsobem definujeme
		$ \Rightarrow^m $, kde $ m \geq 0 $.
		Dále definujeme tranzitivní uzávěr $ \Rightarrow^+ $ a~tranzitivně-reflexivní uzávěr
		$ \Rightarrow^* $.
	\end{definition}

	Algoritmus můžeme uvádět podobně jako definice textově, nebo využít pseudokódu vysázeného ve
	vhodném prostředí (například \texttt{algorithm2e}).

	\begin{algorithm}
		Algoritmus pro ověření bezkontextovosti gramatiky. Mějme gramatiku $ G = (N, T, P, S) $.

		\begin{enumerate}
			\item{
				\label{item_alg_pro_bezkontextovost_gramatiky_1}
				Pro každé pravidlo $ p \in P $ proveď test, zda $ p $ na levé straně obsahuje právě jeden
				symbol z~$ N $.
			}

			\item{
				Pokud všechna pravidla splňují podmínku z~kroku
				\ref{item_alg_pro_bezkontextovost_gramatiky_1}, tak je gramatika $ G $ bezkontextová.
			}
		\end{enumerate}
	\end{algorithm}

	\begin{definition}
		\emph{Jazyk} definovaný gramatikou $ G $ definujeme jako $ L(G) = \{ w \in T^* | S \Rightarrow^* w \} $.
	\end{definition}

	\subsection{Podsekce obsahující větu}

	\begin{definition}
		Nechť $ L $ je libovolný jazyk. $ L $ je \emph{bezkontextový jazyk}, když a~jen když $ L = L(G) $,
		kde $ G $ je libovolná bezkontextová gramatika.
	\end{definition}

	\begin{definition}
		Množinu $ \mathcal{L}_{CF} = \{ L | L $ je bezkontextový jazyk$ \} $ nazýváme
		\emph{třídou bezkontextových jazyků}.
	\end{definition}

	\begin{sentence}
		\label{sentence_1}
		Nechť $ L_{abc} = \{ a^n b^n c^n | n \geq 0 \} $. Platí, že $ L_{abc} \notin \mathcal{L}_{CF} $.
	\end{sentence}

	\begin{proof}
		Důkaz se provede pomocí Pumping lemma pro bezkontextové jazyky, kdy ukážeme, že není možné,
		aby platilo, což bude implikovat pravdivost věty \ref{sentence_1}.
	\end{proof}


	\section{Rovnice a odkazy}

	Složitější matematické formulace sázíme mimo plynulý text. Lze umístit několik výrazů na jeden
	řádek, ale pak je třeba tyto vhodně oddělit, například příkazem \verb|\quad|.

	$$
		\sqrt[x^2]{y^3_0}
		\quad
		\mathbb{N} = \{ 0, 1, 2, \ldots \}
		\quad
		x^{y^y} \neq x^{yy}
		\quad
		z_{i_j} \not\equiv z_{ij}
	$$

	V rovnici (\ref{eq_1}) jsou využity tři typy závorek s~různou explicitně definovanou velikostí.

	\begin{eqnarray}
		\label{eq_1} \bigg\{ \Big[\big(a + b\big) * c\Big]^d + 1 \bigg\} & = & x \\
		\lim_{x\to\infty} \frac{\sin^2x + \cos^2x}{4} & = & y \nonumber
	\end{eqnarray}

	V~této větě vidíme, jak vypadá implicitní vysázení limity $ \lim_{n\to\infty} f(n) $ v~normálním
	odstavci textu. Podobně je to i~s~dalšími symboly jako $ \sum^n_1 $ či
	$ \bigcup_{A \in \mathcal{B}} $. V~případě vzorce $ \lim\limits_{x\to0} \frac{\sin x}{x} = 1 $
	jsme si vynutili méně úspornou sazbu příkazem \verb|\limits|.

	\begin{eqnarray}
		\int\limits^b_a f(x) \, \mathrm{d}x & = & - \int^a_b f(x) \, \mathrm{d}x \\
		\Big(\sqrt[5]{x^4}\Big)^\prime = \Big(x^\frac{4}{5}\Big)^\prime & = & \frac{4}{5} x^{- \frac{1}{5}} = \frac{4}{5 \sqrt[5]{x}} \\
		\overline{\overline{A \vee B}} & = & \overline{\overline{A} \wedge \overline{B}}
	\end{eqnarray}


	\section{Matice}

	Pro sázení matic se velmi často používá prostředí \texttt{array} a~závorky (\verb|\left|, \verb|\right|).

	$$
		\left(
		\begin{array}{cc}
			a + b & b - a \\
			\widehat{\xi + \omega} & \hat{\pi} \\
			\vec a & \overleftrightarrow{AC}\\
			0 & \beta
		\end{array}
		\right)
	$$

	$$
		\mathbf{A} =
		\left\|
		\begin{array}{cccc}
			a_{11} & a_{12} & \ldots & a_{1n} \\
			a_{21} & a_{22} & \ldots & a_{2n} \\
			\vdots & \vdots & \ddots & \vdots \\
			a_{m1} & a_{m2} & \ldots & a_{mn}
		\end{array}
		\right\|
	$$

	$$
		\left|
		\begin{array}{cc}
			t & u \\
			v & w
		\end{array}
		\right|
		= tw - uv
	$$

	Prostředí \texttt{array} lze úspěšně využít i~jinde.

	$$
		\binom{n}{k} =
		\left\{
		\begin{array}{ll}
			\frac{n!}{k! (n - k)!} & \text{pro } 0 \leq k \leq n \\
			0 & \text{pro } k < 0 \text{ nebo } k > n
		\end{array}
		\right.
	$$


	\section{Závěrem}

	V~případě, že budete potřebovat vyjádřit matematickou konstrukci nebo symbol a~nebude se Vám dařit
	jej nalézt v~samotném {\LaTeX}u, doporučuji prostudovat možnosti balíku maker \AmS-\LaTeX. Analogická
	poučka platí obec\-ně pro jakoukoli konstrukci v~{\TeX}u.

\end{document}
